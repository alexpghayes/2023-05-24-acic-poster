% Gemini theme
% https://github.com/anishathalye/gemini

\documentclass[final]{beamer}

% ====================
% Packages
% ====================

\usepackage[T1]{fontenc}
\usepackage{lmodern}
\usepackage[size=custom,width=120,height=72,scale=1.0]{beamerposter}
\usetheme{gemini}
\usecolortheme{mit}
\usepackage{graphicx}
\usepackage{booktabs}

% \usepackage{amsmath}
% \usepackage{amsfonts}
% \usepackage{amsthm}
\usepackage{tikz}
\usepackage{subcaption}
\usepackage{pgfplots}
\pgfplotsset{compat=1.14}
\usepackage{anyfontsize}
\usepackage{hyperref}

\usepackage{hayesmacros}


% \theoremstyle{definition}
% \newtheorem{definition}{Definition}
% \newtheorem{assumption}{Assumption}
% \newtheorem{example}{Example}

% \theoremstyle{plain}
% \newtheorem{proposition}{Proposition}
% \newtheorem{lemma}{Lemma}
% \newtheorem{theorem}{Theorem}
% \newtheorem{corollary}{Corollary}

% \theoremstyle{remark}
% \newtheorem{remark}{Remark}

\hypersetup{colorlinks,allcolors=blue}
\geometry{left=1in,right=1in,top=1in,bottom=1in}

% ====================
% Lengths
% ====================

% If you have N columns, choose \sepwidth and \colwidth such that
% (N+1)*\sepwidth + N*\colwidth = \paperwidth
\newlength{\sepwidth}
\newlength{\colwidth}
\setlength{\sepwidth}{0.025\paperwidth}
\setlength{\colwidth}{0.3\paperwidth}

\newcommand{\separatorcolumn}{\begin{column}{\sepwidth}\end{column}}

% ====================
% Title
% ====================

\title{Estimating network-mediated causal effects via spectral embeddings}

\author{Alex Hayes \inst{1} \and Mark M. Fredrickson \inst{2} \and Keith Levin \inst{1}}

\institute[shortinst]{\inst{1} University of Wisconsin-Madison \samelineand \inst{2} University of Michigan}

% ====================
% Footer (optional)
% ====================

\footercontent{
  \href{https://www.alexpghayes.com}{https://www.alexpghayes.com} \hfill
  ACIC 2023, Austin, TX \hfill
  \href{mailto:alex.hayes@wisc.edu}{alex.hayes@wisc.edu}}
% (can be left out to remove footer)

% ====================
% Logo (optional)
% ====================

% use this to include logos on the left and/or right side of the header:
% \logoright{\includegraphics[height=7cm]{logo1.pdf}}
% \logoleft{\includegraphics[height=7cm]{logo2.pdf}}

% ====================
% Body
% ====================

\begin{document}

\begin{frame}[t]
\begin{columns}[t]
\separatorcolumn

\begin{column}{\colwidth}

  \begin{alertblock}{Abstract}

    Causal inference for network data is an area of active interest in the social sciences. Unfortunately, the complicated dependence structure of network data presents an obstacle to many causal inference procedures. We consider the task of mediation analysis for network data, and present a model in which mediation occurs in a latent embedding space. Under this model, node-level interventions have causal effects on nodal outcomes, and these effects can be partitioned into a direct effect independent of the network, and an indirect effect induced by homophily. To estimate network-mediated effects, we embed nodes into a low-dimensional space and fit two regression models: (1) an outcome model describing how nodal outcomes vary with treatment, controls, and position in latent space; and (2) a mediator model describing how latent positions vary with treatment and controls. We prove that the estimated coefficients are asymptotically normal about the true coefficients under a sub-gamma generalization of the random dot product graph, a widely-used latent space model. We show that these coefficients can be used in product-of-coefficients estimators for causal inference. Our method is easy to implement, scales to networks with millions of edges, and can be extended to accommodate a variety of structured data.

\end{alertblock}

\begin{block}{Motivating example: smoking in adolescent social networks}

    \begin{figure}[ht!]
      \begin{subfigure}{0.49\textwidth}
          \centering
          \includegraphics[width=\textwidth]{figures/glasgow/sex.png}
      \end{subfigure}
      \begin{subfigure}{0.49\textwidth}
          \centering
          \includegraphics[width=\textwidth]{figures/glasgow/tobacco.png}
      \end{subfigure}
      \caption{Directed friendships in a secondary school in Glasgow, reported in the Teenage Friends and Lifestyle Study (wave 1). Each node represents one student. An arrow from node $i$ to node $j$ indicates student $i$ claimed student $j$ as a friend. Node size is proportional to in-degree. On the left, the network is colored by sex. On the right, the network is colored by self-reported smoking frequency.}
      \label{fig:glasgow}
  \end{figure}
  
\end{block}


\begin{block}{Notation \& inferential targets}

  We assume we have a (symmetric) network with nodes $1, ..., n$.

  \begin{table}[]
    \begin{tabular}{lcc}
    Network   & $A$                    & $\R^{n \times n}$ \\
    Treatment & $T_i$                & $\set{0, 1} $               \\
    Outcome   & $Y_i$                & $\R$                \\
    Confounders & $\C_{i \cdot}$ &  $\R^p$                \\
    Friend group & $\X_{i \cdot}$ & $\R^d$               
    \end{tabular}
  \end{table}

  The \emph{average treatment effect} $\ate$ describes how much the outcome $Y_i$ would change on average if the treatment $T_i$ were changed from $T_i = t$ to $T_i = t^*$:
  \begin{equation*}
      \atef = \E{Y_i(t) - Y_i(t^*)}.
  \end{equation*}
  The \emph{natural direct effect} describes how much the outcome $Y_i$ would change if the exposure $T_i$ were set at level $T_i = t^*$ versus $T_i = t$ but for each individual the mediator $\X_{i \cdot}$ were kept at the level it would have taken for that individual, had $T_i$ been set to $t^*$:
  \begin{equation*}
      \ndef = \E{Y_i(t, \X_{i \cdot}(t^*)) - Y_i(t^*, \X_{i \cdot}(t^*))},
  \end{equation*}
  The \emph{natural indirect effect} describes how much the outcome $Y_i$ would change on average if the exposure were fixed at level $T_i = t^*$ but the mediator $\X_{i \cdot}$ were changed from the level it would take under $T_i = t$ to the level it would take under $T_i = t^*$
  \begin{equation*}
      \nief = \E{Y_i(t, \X_{i \cdot}(t)) - Y_i(t, \X_{i \cdot}(t^*))},
  \end{equation*}

\end{block}

\end{column}

\separatorcolumn

\begin{column}{\colwidth}

  \begin{block}{Structural causal model}

    \begin{figure}[ht]
      \centering
      \includegraphics[width=0.6\textwidth]{figures/dags/full_mediating.png}
      \caption{A directed acyclic graph (DAG) representing the causal pathways in a network with homophilous mediation, for node a network with two nodes called $i$ and $j$. We are interested in the causal effect of $T_i$ on $Y_i$ as mediated by the latent position $\X_{i \cdot}$.}
      \label{fig:mediating}
    \end{figure}

  \end{block}

  \begin{block}{Semi-parametric network model}

    Let $A \in \R^{n \times n}$ be a random symmetric matrix, such as the adjacency matrix of an undirected graph. Let $\Apop = \E[\X]{A} = \X \X^T$ be the expectation of $A$ conditional on $\X \in \R^{n \times d}$, which has independent and identically distributed rows $\X_{1 \cdot}, \dots, \X_{n \cdot}$. That is, $\Apop$ has $\rank \paren*{\Apop} = d$ and is positive semi-definite with eigenvalues $\lambda_1 \ge \lambda_2 \ge \cdots \ge \lambda_d > 0 = \lambda_{d+1} = \cdots = \lambda_n$. Conditional on $\X$, the upper-triangular elements of $A - \Apop$ are independent $(\nu_n, b_n)$-sub-gamma random variables.
    
    The outcome regression functional is linear in $T_i, \C_{i \cdot}$, and $\X_{i \cdot}$ and the mediator regression functional is linear in $T_i, \C_{i \cdot}$, and $T_i \cdot \C_{i \cdot}$:
    
    \begin{equation*} 
      \begin{aligned}
            \underbrace{\E[T_i, \C_{i \cdot}, \X_{i \cdot}]{Y_i}}_{\R}
              & = \underbrace{\betazero}_{\R}
            + \underbrace{T_i}_{\{0, 1\}} \underbrace{\betat}_{\R}
            + \underbrace{\C_{i \cdot}}_{\R^{1 \times p}} \underbrace{\betac}_{\R^{p}}
            + \underbrace{\X_{i \cdot}}_{\R^{1 \times d}} \underbrace{\betax}_{\R^d},
              & \text{(outcome model)}                      \\
            \underbrace{\E[T_i, \C_{i \cdot}]{\X_{i \cdot}}}_{\R^{1 \times d}}
              & = \underbrace{\thetazero}_{\R^{1 \times d}}
            + \underbrace{T_i}_{\{0, 1\}} \underbrace{\thetat}_{\R^{1 \times d}}
            + \underbrace{\C_{i \cdot}}_{\R^{1 \times p}} \underbrace{\Thetac}_{\R^{p \times d}}
            + \underbrace{T_i}_{\{0, 1\}} \underbrace{\C_{i \cdot}}_{\R^{1 \times p}} \underbrace{\Thetatc}_{\R^{p \times d}}.
              & \text{(mediator model)}
      \end{aligned}
    \end{equation*}
  \end{block}

  \begin{block}{Semi-parametric causal identification}

    
    Let $\mu_c$ denote the mean of $\C_{i \cdot}$. Then,
    \begin{align*}
        \ndef & = \paren*{t - t^*} \, \betat, ~ \text{ and }                                    \\
        \nief & = \paren*{t - t^*} \, \thetat \, \betax + (t - t^*) \, \mu_c \, \Thetatc \, \betax. 
    \end{align*}

  \end{block}

  \begin{block}{Estimation}
    Given a network with adjacency matrix $A$, the $d$-dimensional \emph{adjacency spectral embedding} (ASE) of $A$ is defined as
    \begin{equation*}
        \Xhat = \Uhat \Shat^{1/2} \in \R^{n \times d},
    \end{equation*}
    where $\Uhat \Shat \Uhat^T$ is the rank-$d$ truncated singular value decomposition of $A$. That is, $\Shat \in \R^{d \times d}$ is diagonal, with entries given by the $d$ leading singular values of $A$, and $\Uhat \in \R^{n \times d}$ has the corresponding $d$ orthonormal singular vectors as its columns.

    Let $\W = \begin{bmatrix} 1 & T & \C \end{bmatrix} \in \R^{n \times (p + 2)}$ and $\Wfull = \begin{bmatrix} W & T \cdot \C \end{bmatrix} \in \R^{n \times (2 p + 2)}$.

    Define $\Dhat = \begin{bmatrix} \W & \Xhat \end{bmatrix} \in \R^{n \times (2 + p + d)}$. We estimate $\betaw$ and $\betax$ via ordinary least squares as follows
    \begin{equation*}
        \begin{bmatrix}
            \betawhat \\
            \betaxhat
        \end{bmatrix}
        = \paren*{\Dhat^T \Dhat}^{-1} \Dhat^T Y.
    \end{equation*}

    % Similarly, we estimate $\Theta$ via ordinary least squares as
    % \begin{equation*}
    %     \Thetahat
    %     = \paren*{\Wfull^T \Wfull}^{-1} \Wfull^T \Xhat.
    % \end{equation*}
    
  \end{block}

\end{column}

\separatorcolumn

\begin{column}{\colwidth}

  \begin{exampleblock}{A highlighted block containing some math}

    A different kind of highlighted block.

    $$
    \int_{-\infty}^{\infty} e^{-x^2}\,dx = \sqrt{\pi}
    $$

    Interdum et malesuada fames $\{1, 4, 9, \ldots\}$ ac ante ipsum primis in
    faucibus. Cras eleifend dolor eu nulla suscipit suscipit. Sed lobortis non
    felis id vulputate.

    \heading{A heading inside a block}

    Praesent consectetur mi $x^2 + y^2$ metus, nec vestibulum justo viverra
    nec. Proin eget nulla pretium, egestas magna aliquam, mollis neque. Vivamus
    dictum $\mathbf{u}^\intercal\mathbf{v}$ sagittis odio, vel porta erat
    congue sed. Maecenas ut dolor quis arcu auctor porttitor.

    \heading{Another heading inside a block}

    Sed augue erat, scelerisque a purus ultricies, placerat porttitor neque.
    Donec $P(y \mid x)$ fermentum consectetur $\nabla_x P(y \mid x)$ sapien
    sagittis egestas. Duis eget leo euismod nunc viverra imperdiet nec id
    justo.

  \end{exampleblock}

  \begin{block}{Nullam vel erat at velit convallis laoreet}

    Class aptent taciti sociosqu ad litora torquent per conubia nostra, per
    inceptos himenaeos. Phasellus libero enim, gravida sed erat sit amet,
    scelerisque congue diam. Fusce dapibus dui ut augue pulvinar iaculis.

    \begin{table}
      \centering
      \begin{tabular}{l r r c}
        \toprule
        \textbf{First column} & \textbf{Second column} & \textbf{Third column} & \textbf{Fourth} \\
        \midrule
        Foo & 13.37 & 384,394 & $\alpha$ \\
        Bar & 2.17 & 1,392 & $\beta$ \\
        Baz & 3.14 & 83,742 & $\delta$ \\
        Qux & 7.59 & 974 & $\gamma$ \\
        \bottomrule
      \end{tabular}
      \caption{A table caption.}
    \end{table}

    Donec quis posuere ligula. Nunc feugiat elit a mi malesuada consequat. Sed
    imperdiet augue ac nibh aliquet tristique. Aenean eu tortor vulputate,
    eleifend lorem in, dictum urna. Proin auctor ante in augue tincidunt
    tempor. Proin pellentesque vulputate odio, ac gravida nulla posuere
    efficitur. Aenean at velit vel dolor blandit molestie. Mauris laoreet
    commodo quam, non luctus nibh ullamcorper in. Class aptent taciti sociosqu
    ad litora torquent per conubia nostra, per inceptos himenaeos.

    Nulla varius finibus volutpat. Mauris molestie lorem tincidunt, iaculis
    libero at, gravida ante. Phasellus at felis eu neque suscipit suscipit.
    Integer ullamcorper, dui nec pretium ornare, urna dolor consequat libero,
    in feugiat elit lorem euismod lacus. Pellentesque sit amet dolor mollis,
    auctor urna non, tempus sem.

  \end{block}

  \begin{block}{References}

    This talk is based on a manuscript recently submitted to JRSS-B, available as a pre-print at https://arxiv.org/abs/2212.12041.

    \nocite{*}
    \footnotesize{\bibliographystyle{plain}\bibliography{poster}}

  \end{block}

\end{column}

\separatorcolumn
\end{columns}
\end{frame}

\end{document}
